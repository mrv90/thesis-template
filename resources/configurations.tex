% فایل اصلی پیکربندی؛ بسته‌ها و فونت‌های استفاده شده، عموما در این قسمت قرار می‌گیرد.

\documentclass[a4paper,10pt]{report}

\usepackage{amssymb}% http://ctan.org/pkg/amssymb برای استفاده از سمبل‌ها و کاراکتر‌های خاص
\usepackage{pifont}% http://ctan.org/pkg/pifont برای استفاده از سمبل‌ها و کاراکتر‌های خاص

\usepackage{lipsum}% برای استفاده از noindent

\usepackage{ragged2e} % برای استفاده از جاستیفای
\usepackage[nonamebreak,square]{natbib} % استایل منابع به این بسته نیاز دارد
\usepackage{float} % برای ثابت نگه‌داشتن مکان تصاویر
\usepackage{titlesec} % برای تغییر فونت عناوین مختلف
\usepackage[top=25mm, bottom=25mm, left=25mm, right=35mm, includehead=true, includefoot=true]{geometry} % ابعاد
\usepackage{tabularx} % برای تعریف مجدد بعضی دستورات جدول
\usepackage{fontspec}
\usepackage[nottoc]{tocbibind} % نمایش کتابنامه در فهرست؛ اما خود فهرست نباید درفهرست باشد
\usepackage[usenames,dvipsnames]{color}
\usepackage[table]{xcolor}% برای رنگ‌های جداول
\usepackage{listings} % برای حروف چینی کد برنامه‌سازی
\usepackage{todonotes} % برای ایجاد یادداشت‌های مشابه با استیکی‌نوت
\usepackage{graphicx}
\usepackage{caption}
\usepackage{hyperref}
%\usepackage[toc,xindy,acronym,nonumberlist=true]{glossaries}

\usepackage[localise=on,extrafootnotefeatures]{xepersian} % بسته اصلی و الزامی که باید آخرین بسته (در ترتیب استفاده) باشد
\settextfont[Scale=1.3]{B Lotus} % 13pt %{XB Niloofar} (فونت اصلی؛ (نیلوفر بسیار به لوتوس شبیه است
\setlatintextfont[Scale=1.1]{Times New Roman} % 11pt main latin font
% باقی فونت‌های استفاده شده. در جاهایی که شیوه‌نامه، انتخاب فونت را دیکته نکرده، به صورت دلخواه <<فونت تعریف شده>>، در اندازه مورد نیاز، استفاده شد.
\defpersianfont\nst[Scale=5.0]{IranNastaliq} 
\defpersianfont\bn[Scale=1.0]{Bahij Nazanin}
\defpersianfont\nil[Scale=1.0]{XB Niloofar}
\deflatinfont\tnr[Scale=1.0]{Times New Roman} 
\deflatinfont\cnt[Scale=1.0]{Cantarell} % مناسب برای نوشتن کد برنامه‌سازی
\defpersianfont\shn[Scale=1.0]{Shekasteh_Beta} % شکسته نستعلیق
\defpersianfont\bsm[Scale=1.0]{S 110_Besmellah}
\defpersianfont\msh[Scale=1.0]{B Mashhad}

\defpersianfont\nkh[Scale=1.0]{Adobe Naskh} % فونت زبان عربی
\defpersianfont\bst[Scale=1.0]{Bustani} % فونت زبان عربی
\defpersianfont\din[Scale=1.0]{DINArabic} % فونت زبان عربی


\linespread{1.5} % فاصله بین خطوط :‌ معادل با ۱ سانتی متر

\renewcommand\thefigure{\thechapter-\arabic{figure}} % نقطه موجود در نام شکل، با خط تیره جا‌به‌جا شود
\renewcommand\thetable{\thechapter-\arabic{table}} % نقطه موجود در نام جدول، با خط تیره جابه‌جا شود

