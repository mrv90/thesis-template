% دستورات جدید و دستوراتی که برای اصلاح، بازتعریف شدند، در این فایل قرار دارند؛ بنابراین، مشابه با توابع/متد‌های زبان‌های برنامه‌سازی، با فراخوانی نام دستور و پاس دادن پارامترها (دقیقا به تعداد موردنیاز) میتوان از این دستورات استفاده کرد.

\newcommand{\inTheNameOfGod}
{
  \begin{center}
    \vspace*{\fill}
    \begin{quote}
      \centering
      %\nst بسم الله الرّحمن الرّحیم
      {\shn\fontsize{60}{72}\selectfont بسم الله الرّحمن الرّحیم}
    \end{quote}
    \vspace*{\fill}
  \end{center}
}
\newcommand{\blankpage}
{
  \pagenumbering{gobble} % صفحه خالی، شماره صفحه ندارد
  \clearpage
   \vspace*{\fill}
     \begin{center}
       %{\nst\fontsize{4}{5.8}\selectfont[این صفحه آگاهانه خالی گذاشته شده است.]}
       \textit{\nil\fontsize{12}{14.4}\selectfont \textbf[این صفحه آگاهانه خالی گذاشته شده است.]}
     \end{center}
   \vspace*{\fill}
  \clearpage
  %\pagenumbering{myabjad} % شماره گزاری بروش ابجد شروع شود
}

\newcommand{\farpnulogo}[1] % نام واحد / مرکز
{
  \clearpage % در شیوه نامه،قسمت جلد فارسی،‌اولین قسمت لوگو دانشگاه است
  \begin{figure}
    \centering
    \includegraphics[scale=0.8]{resources/img/pnu_logo-colored.png}
    \caption*{{\nst\fontsize{4}{4.8}\selectfont \textbf{دانشگاه پیام‌نور}}}
    {\nst\fontsize{3}{3.6}\selectfont \textbf{#1}}
    %\mbox{#1}
    %\label{fig0}
  \end{figure}
  \bigskip
}

\newcommand{\farfieldofstudy}[2] % نام رشته تحصیلی و گرایش
{
  \begin{center}
    {\bn\fontsize{14}{16.8}\selectfont \textbf{\rl{پایان‌نامه کارشناسی‌ارشد رشته #1}}}\\
    {\bn\fontsize{14}{16.8}\selectfont \textbf{\rl{گرایش: #2}}}
    \bigskip
    \bigskip
  \end{center}
}

\newcommand{\farthesistitle}[1]
{
  \begin{center}
    {\bn\fontsize{15}{18}\selectfont عنوان پایان‌نامه:}\\
    {\bn\fontsize{20}{24}\selectfont #1}
    \bigskip
    \bigskip
  \end{center}
}

\newcommand{\farsupervisor}[1]
{
  \begin{center}
    {\bn\fontsize{14}{16.8}\selectfont استاد راهنما:}\\
    {\bn\fontsize{16}{19.2}\selectfont \textbf{#1}}
    \bigskip
    \bigskip
  \end{center}
}

\newcommand{\faradvisor}[1]
{
  \begin{center}
    {\bn\fontsize{14}{16.8}\selectfont استاد مشاور:}\\
    {\bn\fontsize{16}{19.2}\selectfont \textbf{#1}}
    \bigskip
    \bigskip
  \end{center}
}

\newcommand{\farauthorofthesis}[1]
{
  \begin{center}
    {\bn\fontsize{14}{16.8}\selectfont نگارنده:}\\
    {\bn\fontsize{16}{19.2}\selectfont \textbf{#1}}
    \bigskip
    \bigskip
  \end{center}
}

\newcommand{\fardateofdefence}[1]
{
  \begin{center}
    {\bn\fontsize{14}{16.8}\selectfont \textbf \rl{#1}}
  \end{center}
  \clearpage % در شیوه نامه، قسمت جلد فارسی، آخرین قسمت زمان دفاع است
}

\newcommand{\juryapproval}[1]
{
  \centering{\fontsize{15}{18}\selectfont \textbf{\underline{تصویب‌نامه}}}\\
  \bigskip
  \centering{\fontsize{12}{14.4}\selectfont {پایان‌نامه کارشناسی‌‌ارشد رشته #1}}\\
}

\newcommand{\authorofthesisinline}[1]
{
  \centering{\fontsize{12}{14.4}\selectfont \rl{#1}}
}

\newcommand{\thesistitleinline}[1]
{
  \flushright{\fontsize{12}{14.4}\selectfont \textbf{تحت عنوان:}}\\
  \justify{\fontsize{12}{14.4}\selectfont #1}\\
}

\newcommand{\dateofdefense}[2]
{
  %\noindent{\rl{تاریخ دفاع: #1}{\rl{ساعت: #2}}
  \flushright{\rl{تاریخ دفاع: #1}} \rl{ساعت: #2}\\
}

\newcommand{\thesisgrade}
{
   \bigskip   
   \centering{\fontsize{12}{14.4}\selectfont {نمره: ..........}}\\
   \bigskip
   \centering{\fontsize{12}{14.4}\selectfont {درجه ارزشیابی: ..........}}\\
}

\newcommand{\examiners}[8]
{
  \flushright{\large {هیات داوران:}}
  \begin{table}[H]
  \centering
  \def\arraystretch{1.7}%  horzintal spacing; 1 is the default, change whatever you need
  \begin{tabular}{|c|c|c|C{4cm}|}
  \hline
   \textbf{داوران} & \textbf{نام‌و‌نام‌خانوادگی} & \textbf{مرتبه‌علمی} & \textbf{امضا}\\
  \hline
   \rl{\small \textbf{استاد راهنما}} & \rl{\normalsize #1} & \rl{\small #2} & \\
   \rl{\small \textbf{استاد مشاور}} & \rl{\normalsize #3} & \rl{\small #4} & \\
   \rl{\small \textbf{استاد داور}} & \rl{\normalsize #5} & \rl{\small #6} & \\
   \rl{\small \textbf{نماینده تحصیلات تکمیلی}} & \rl{\normalsize #7} & \rl{\small #8} & \\
  \hline
  \end{tabular}
  \end{table}
  \flushright{\fontsize{12}{14.4}\selectfont {آیا پایان‌نامه مذکور نیاز به اصلاحات دارد؟}}
}

\newcommand{\commitment}[5]
{
  \vspace*{\stretch{1}}
  \clearpage 
  \begin{center}
  {\fontsize{16}{19.2}\selectfont \textbf{تعهد پایان‌نامه}}\\
  \bigskip
  \bigskip 
  \bigskip
  \justify{اینجانب #1 دانش‌آموخته مقطع کارشناسی‌ارشد ناپیوسته تخصصی در رشته #2 که در تاریخ #3 از پایان‌نامه خود با عنوان #4 با کسب نمره #5 دفاع نموده‌ام، بدینوسیله متعهد می‌شوم:}\\
  %\bigskip
  \begin{enumerate}
\item این پایان‌نامه حاصل تحقیق و پژوهش انجام شده توسط اینجانب بوده و در مواردی که از دستاورد‌های عملی و پژوهش دیگران (اعم از پایان‌نامه، کتاب، مقاله {و...}) استفاده نموده‌ام مطابق ضوابط و رویه موجود، نام منبع مورد استفاده و سایر مشخصات آن را در فهرست مربوط ذکر ودرج کرده‌ام.\\
  \bigskip
\item این پایان‌نامه قبلا برای دریافت هیچ مدرک تحصیلی (هم سطح، پایین‌تر یا بالاتر) در سایر دانشگاه‌ها و موسسات آموزش عالی ارائه نشده است.\\

\item چناچه بعد از فراغت از تحصیل،‌ قصد استفاده و هرگونه بهره‌برداری اعم از چاپ مقاله، کتاب، ثبت اختراع {و...} از این پایان‌نامه را داشته باشم، با ذکر نام استادان راهنما و مشاور و درج نام دانشگاه پیام‌نور اقدام خواهم کرد. \\

\item چنانچه در هر مقطع زمانی خلاف موارد فوق ثابت شود، عواقب حقوقی ناشی از آن را می‌پذیرم و همچنین دانشگاه پیام‌نور مجاز است با اینجانب مطابق ضوابط و مقررات رفتار کند و در صورت ابطال مدرک تحصیلی‌ام هیچ گونه ادعایی نخواهم داشت.\\
  \end{enumerate}
  \centering\textbf{نام و نام‌خانوادگی}\\
  \bigskip
  \centering ....................\\
  \bigskip\bigskip
  \centering\textbf{تاریخ و امضا}\\
  \centering ....................\\
  \bigskip\bigskip\bigskip\bigskip\bigskip
  \color{red}\underline{این تعهدنامه باید پس از امضا در متن پایان نامه درج گردد.}
  \end{center}
  \vspace*{\stretch{1}}
  \clearpage 
}

\newcommand{\engpnulogo}
{
  \clearpage % در شیوه نامه،قسمت جلد انگلیسی،‌اولین قسمت لوگو دانشگاه است
  \begin{figure}
    \centering
    \includegraphics[scale=0.8]{resources/img/pnu_logo-colored.png}
    \caption*{{\tnr \fontsize{16}{19.2}\selectfont \textbf{\lr{Payame Noor University}}}}
  \end{figure}
  \bigskip
}

\newcommand{\engdepartment}[2] % نام دانشکده
{
  \begin{center}
    {\tnr \fontsize{14}{16.8}\selectfont \textbf{\lr{Department of #1}}}
    \bigskip
  \end{center}
}

\newcommand{\enginterest}[1]
{
  \begin{center}
    {\tnr \fontsize{14}{16.8}\selectfont \textbf{\lr{Thesis Submitted in Partial Fulfillment 
of the requirement for the Degree of M.Sc In #1}}}
  \end{center}
}

\newcommand{\engthesistitle}[1]
{
  \begin{center}
    {\tnr \fontsize{14}{16.8}\selectfont \textbf{\lr{Title:}}}\\
    {\tnr \fontsize{20}{24}\selectfont \textbf{\lr{#1}}}
    \bigskip
  \end{center}
}

\newcommand{\engsupervisor}[1]
{
  \begin{center}
    {\tnr \fontsize{14}{16.8}\selectfont \textbf{\lr{Superviser:}}}\\
    {\tnr \fontsize{16}{19.2}\selectfont \textbf{\lr{Dr. #1}}}
    \bigskip
  \end{center}
}

\newcommand{\engadvisor}[1]
{
  \begin{center}
    {\tnr \fontsize{14}{16.8}\selectfont \textbf{\lr{Advisor:}}}\\
    {\tnr \fontsize{16}{19.2}\selectfont \textbf{\lr{Dr. #1}}}
    \bigskip
  \end{center}
}

\newcommand{\engauthorofthesis}[1]
{
  \begin{center}
    {\tnr \fontsize{14}{16.8}\selectfont \textbf{\lr{By:}}}\\
    {\tnr \fontsize{16}{19.2}\selectfont \textbf{\lr{#1}}}
    \bigskip
  \end{center}
}

\newcommand{\engdateofdefence}[1]
{
  \begin{center}
    {\tnr \fontsize{14}{16.8}\selectfont \textbf{\lr{#1}}}
  \end{center}
  \clearpage % در شیوه نامه، قسمت جلد فارسی، آخرین قسمت زمان دفاع است
}

\newcommand{\gratitude}[8]
{
   {\bn\fontsize{16}{19.2}\selectfont \textbf{#1}}
   \bigskip
   \bigskip\\
   {\bn\fontsize{16}{19.2}\selectfont {#2}}
   \bigskip\\
   {\bn\fontsize{16}{19.2}\selectfont {#3}}
   \bigskip\\
   \begin{center}
      {\nst\fontsize{8}{9.6}\selectfont {#4}}
   \end{center}
   \bigskip
   {\bn\fontsize{16}{19.2}\selectfont {#5}}
   \bigskip\\
   \begin{center}
      {\nst\fontsize{8}{9.6}\selectfont {#6}}
   \end{center}
   \bigskip
   {\bn\fontsize{16}{19.2}\selectfont {#7}}
   \bigskip
   \bigskip\\
   {\bn\fontsize{16}{19.2}\selectfont {#8}}
   \clearpage
}

\newcommand{\dedications}[4]
{
   \begin{flushright}
     {\bn \fontsize{20}{24}\selectfont {#1}}\\
     \bigskip
     \bigskip
     {\nst \fontsize{7}{9.8}\selectfont {#2}}\\
     {\nst \fontsize{7}{9.8}\selectfont {#3}}\\
     {\nst \fontsize{7}{9.8}\selectfont {#4}}\\
     \bigskip
     \clearpage
   \end{flushright}
}

\newcommand{\bottomquote}[3]
{
   \vspace*{\fill}
   \begin{flushleft}
     {\nkh \fontsize{25}{30}\selectfont \mbox{#1}}\\
     \bigskip
     {\shn \fontsize{25}{30}\selectfont \mbox{#2}}\\
     \bigskip
     {\bn \fontsize{12}{14.2}\selectfont \mbox{#3}}\\
   \end{flushleft}
}
% ================================================================================
% شماره گزاری ابجد صفحات
\makeatletter
\def\@myabjad#1{\ifcase#1\or الف\or ب\or ج\or د\or ه\or
و\or ز\or ح\or ط\or ی\or ک\or ل\or م\or ن\or س\or ع\or ف\or ص\or غ\or
ر\or ش\or ت\or ث\
else\@ctrerr\fi}
\makeatother

% ================================================================================
% دستوری برای تعریف واژه‌نامه انگلیسی به فارسی
\newcommand\englishgloss[2]{#1\dotfill\lr{#2}\\}
% دستوری برای تعریف واژه‌نامه فارسی به انگلیسی 
\newcommand\persiangloss[2]{#2\dotfill\lr{#1}\\}
% ================================================================================
% استایل و نحوه نمایش خطوط کد سی و سی‌پلاس‌پلاس (در پیوست‌ها)

\lstdefinestyle{customc}
{
  belowcaptionskip=1\baselineskip,
  breaklines=true,
  frame=L,
  xleftmargin=\parindent,
  language=C,
  showstringspaces=false,
  basicstyle=\cnt \fontsize{9}{10.8}\selectfont,
  keywordstyle=\bfseries\color{green!40!black},
  commentstyle=\itshape\color{purple!40!black},
  identifierstyle=\color{blue},
  stringstyle=\color{orange},
}

\lstdefinestyle{customcpp}
{
  belowcaptionskip=1\baselineskip,
  breaklines=true,
  frame=L,
  xleftmargin=\parindent,
  language=C++,
  morekeywords={TEST_F,ASSERT_TRUE,ASSERT_EQ,std::string,std::cin,std::cout,std::ifstream,std::endl},
%  deletekeywords={#include},
%  escapeinside={}{}, % باید include و define از کلمات کلیدی مجزا شوند ؟؟
  showstringspaces=false,
  basicstyle=\cnt \fontsize{9}{10.8}\selectfont, 
  keywordstyle=\bfseries\color{Fuchsia}, %NavyBlue رنگ کلمات بعد از
  commentstyle=\itshape\color{NavyBlue},
  identifierstyle=\color{ForestGreen},
  stringstyle=\color{black},
  numbers=left,
}

% ================================================================================
% مطابق با آیین‌نامه، <<فهرست تصاویر>> مناسب نیست؛‌ <<فهرست شکل‌ها و نمودار‌ها>> قابل قبول است
\renewcommand\listfigurename{فهرست شکل‌ها و نمودارها}  
% مطابق با آیین‌نامه، <<فهرست جداول>> مناسب نیست؛ <<فهرست جدول‌ها>> قابل قبول است
\renewcommand\listtablename{فهرست جدول‌ها}
% مطابق با آیین‌نامه، <<کتاب‌نامه>> مناسب نیست؛ <<فهرست منابع>>‌ جایگزین شد
\renewcommand\bibname{\rl{فهرست منابع}}

% ================================================================================
% تعریف قلم‌های فارسی و انگلیسی اضافی برای استفاده در بعضی از قسمت‌های متن
\defpersianfont\nastaliq[Scale=2]{IranNastaliq}
\defpersianfont\chapternumber[Scale=3]{XB Niloofar}

% دستورهایی برای سفارشی کردن صفحات اول فصل‌ها
\makeatletter
\newcommand\mycustomraggedright{%
 \if@RTL\raggedleft%
 \else\raggedright%
 \fi}
\def\@part[#1]#2{%
\ifnum \c@secnumdepth >-2\relax
\refstepcounter{part}%
\addcontentsline{toc}{part}{\thepart\hspace{1em}#1}%
\else
\addcontentsline{toc}{part}{#1}%
\fi
\markboth{}{}%
{\centering
\interlinepenalty \@M
\ifnum \c@secnumdepth >-2\relax
 \huge\bfseries \partname\nobreakspace\thepart
\par
\vskip 20\p@
\fi
\Huge\bfseries #2\par}%
\@endpart}
\def\@makechapterhead#1{%
\vspace*{-30\p@}%
{\parindent \z@ \mycustomraggedright %\@mycustomfont
\ifnum \c@secnumdepth >\m@ne
\if@mainmatter

\huge\bfseries \@chapapp\space {\chapternumber\thechapter}
\par\nobreak
\vskip 20\p@
\fi
\fi
\interlinepenalty\@M 
\Huge \bfseries #1\par\nobreak
\vskip 120\p@
}}
\makeatother
% ================================================================================
% تغییر قلم عنوان فصل
%\titleformat{\chapter}[display]
%{\nil\huge\bfseries}{\chaptertitlename\ \thechapter}{20}{\Huge}
% ================================================================================
% http://ctan.org/pkg/amssymb
% http://ctan.org/pkg/pifont
\newcommand{\cmark}{\ding{51}}% تیک؛ کد سمبل از پی‌دی‌اِف مرجع استخراج شده
\newcommand{\xmark}{\ding{55}}% زبدر
% ================================================================================
\definecolor{LRed}{rgb}{1,.8,.8}
\definecolor{MRed}{rgb}{1,.6,.6}
\definecolor{HRed}{rgb}{1,.2,.2}

\definecolor{Black}{rgb}{0,0,0}
\definecolor{White}{rgb}{255,255,255}
\definecolor{Red}{rgb}{255,0,0}
\definecolor{Lime}{rgb}{0,255,0}
\definecolor{Blue}{rgb}{0,0,255}
\definecolor{Yellow}{rgb}{255,255,0}
\definecolor{Cyan}{rgb}{0,255,255}
\definecolor{Magenta}{rgb}{255,0,255}
\definecolor{Silver}{rgb}{192,192,192}
\definecolor{Gray}{rgb}{128,128,128}
\definecolor{Maroon}{rgb}{128,0,0}
\definecolor{Olive}{rgb}{128,128,0}
\definecolor{Green}{rgb}{0,128,0}
\definecolor{Purple}{rgb}{128,0,128}
\definecolor{Teal}{rgb}{0,128,128}
\definecolor{Navy}{rgb}{0,0,128}
% ================================================================================
\newcommand\textbox[1]{%
  \parbox{.333\textwidth}{#1}%
}
% ================================================================================
\newcolumntype{L}[1]{>{\raggedright\arraybackslash}p{#1}}
\newcolumntype{C}[1]{>{\centering\arraybackslash}p{#1}}
\newcolumntype{R}[1]{>{\raggedleft\arraybackslash}p{#1}}
